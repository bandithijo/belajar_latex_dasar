\documentclass[12pt, times new roman]{article}
\linespread{1.5}
\usepackage{geometry}
\geometry{a4paper, total={210mm, 297mm}, left=40mm, right=30mm, top=30mm, bottom=30mm}
\begin{document}
Hello teman-teman! Pada kesempatan kali ini, kita akan belajar mengenai cara membuat \textbf{footnote}.

Variable adalah kemungkinan yang minimal memiliki satu kategori \footnote{Shofwan, 2013, Kuantitatif, 1007}. Selain itu, variable juga dibagi menjadi dua macam, yaitu variable bebas dan terikat \footnote{Ibid, 1008}.
\end{document}
