\documentclass{article}
\linespread{1.5}
\usepackage{geometry}
\geometry{a4paper, left=40mm, right=30mm, top=30mm, bottom=30mm}
\usepackage{hyperref}
\renewcommand{\refname}{Daftar Pustaka}
\begin{document}
\section{Pendahuluan}
Variable adalah kemungkinan yang minimal memiliki satu kondisi \cite{Shofwan}. Dua jenis variable adalah bebas dan terikat \cite{Ali, Fauzi}.

BSPWM adalah window manager yang praktis namun mudah digunakan. Tidak memerlukan pekerjaan. Tidak banyak membawa komponen. Hanya file binary dari BSPWM dan aplikasi untuk menghandle inputan keyboard yang namanya cukup sulit untuk dibaca, SXHKD \cite{Rizqi}.

Mencoba fitur vimtex dengan auto compiler.


% Daftar Pustaka
\begin{thebibliography}{3}
  \bibitem{Shofwan}
  Shofwan, 2012, \textit{Kuantitatif}, Penerbit Aku : Surabaya

  \bibitem{Ali}
  Ali, 2014, \textit{Buku-ku Dewe}, Al-Ali : Jember

  \bibitem{Fauzi}
  Fauzi, \texttt{www.kunjungilah.com}

  \bibitem{Rizqi}
  Assyaufi, Rizqi Nur, 2020, \textit{Semua Bisa Menjadi Programmer Ruby}, Gramedia : Jakarta
\end{thebibliography}
\end{document}
