\documentclass[12pt, times new roman]{article}
\linespread{1.5}
\usepackage{geometry}
\geometry{a4paper, total={210mm, 297mm}, left=40mm, right=30mm, top=30mm, bottom=30mm}
\usepackage{caption}
\usepackage{hyperref}
\renewcommand{\tablename}{Tabel}
\begin{document}
Tabel \ref{tab:tabelsatu} di bawah ini adalah contoh hasil dari pembuatan tabel menggunakan minpage pada LaTeX.\\
\begin{center}
  \begin{minipage}{\textwidth}
    \begin{center}
      \captionof{table}{Jumlah mahasiswa menggunakan minipage LaTeX}
      \label{tab:tabelsatu}
      \begin{tabular}{|l|c|c|c|}
        \hline
        Fakultas & \multicolumn{2}{|c|}{Jenis Kelamin} & Jumlah\\
        \cline{2-3} {} & {Laki-laki} & {Perempuan} & {}\\
        \hline
        FTIK & 34 & 56 & \textbf{90}\\
        FUAH & 54 & 36 & \textbf{90}\\
        FEBY & 55 & 45 & \textbf{100}\\
        \hline
      \end{tabular}
    \end{center}
  \end{minipage}
\end{center}
\end{document}
