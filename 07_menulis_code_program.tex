\documentclass[12pt, times new roman]{article}
\usepackage{geometry}
\geometry{a4paper, total={210mm, 297mm}, left=40mm, right=30mm, top=30mm, bottom=30mm}
\linespread{1.5}
\begin{document}
Halo teman-teman,

Hari ini kita akan belajar menulis blok kode di dalam LaTeX. Kita akan menggunakan fungsi \textbf{verbatim}.

\begin{verbatim}
class SubscribersController < ApplicationController
  def create
    subscriber = Subscriber.new(resource_params)

    if subscriber.save
      status = 'ok'
      message = 'Terima kasih dukungannya :)'
    else
      status = 'error'
      message = subscriber.errors[:email].first
      message
    end

    render json: {status: status, message: message}
  end

  private

  def resource_params
    params.require(:subscriber).permit(:email)
  end
end

# Testing the Comment Color
\end{verbatim}

Nah, kira-kira seperti di atas inilah hasilnya. Saya akan mencontohkan lagi untuk kasus \textit{command line} Terminal.

\begin{verbatim}
$ doas pkg install latex
\end{verbatim}

Di atas, adalah contoh untuk perintah di dalam Terminal.
\end{document}
