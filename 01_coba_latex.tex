\documentclass[12pt, Liberation Sans]{article}
\linespread{1.5}
\usepackage{caption}
\usepackage{hyperref}
\renewcommand{\tablename}{Tabel}
\usepackage{geometry}
\geometry{a4paper,
  total = {210mm, 297mm},
  left = 40mm,
  right = 30mm,
  top = 30mm,
bottom = 30mm}
\begin{document}
Setiap hari, ayah saya berangkat ke pasar untuk membeli bahan-bahan keperluan membuat bakso. Ayah saya pulang membeli bahan sekitar pukul 6 pagi. Beliau langsung membuat bakso yang cukup untuk dibuat berjualan selama lima jam. Ayah membutuhkan waktu sepuluh jam untuk membuat bakso. Kemudian beliau tidur siang dan bangun untuk sholat dhuhur.

Setelah \textbf{sholat dhuhur} pergi ke kebun untuk merawat padi. Beliau merawat padi di kebun milik juragan padi yang terkenal di kampung kami. Juragan ini memiliki sawah di mana-mana. Beliau banyak mempekerjakan warga kampung kami.\\

\begin{center}
  \begin{minipage}{\textwidth}
    \captionof{table}{Di bawah ini adalah matriks kedua}
    \label{tab:tabelsatu}
    \begin{center}
      \begin{tabular}{|c|c|}
        \hline
        a & b\\
        \hline
        c & d\\
        \hline
      \end{tabular}
    \end{center}
  \end{minipage}
  \newline
\end{center}

Seperti yang sudah ditunjukkan pada Tabel \ref{tab:tabelsatu} di atas. Kita dapat membuat matrik 2x2 dengan mudah menggunakan LaTex. Hal yang sama juga dapat kita buat untuk membuat tabel biasa.\\

\begin{center}
  \begin{minipage}{\textwidth}
    \captionof{table}{Total donasi dari anggota sepak bola}
    \label{tab:tabeldua}
    \begin{tabular}{|c|l|c|r|r|r|r|}
      \hline
      No. & Nama & NIP & Donasi 1 & Donasi 2 & Donasi 3 & Jumlah Donasi\\
      \hline
      1 & Rizqi Nur Assyaufi & 1511157 & 100.000 & 50.000 & 100.000 & 250.000\\
      \hline
      2 & Budi Sudarsono & 1511146 & 120.000 & 90.000 & 100.000 & 310.000\\
      \hline
      3 & Bambang Pamungkas & 1511130 & 80.000 & 250.000 & 100.000 & 430.000\\
      \hline
      \multicolumn{6}{|l|}{Total Donasi Terkumpul} & \textbf{690.000}\\
      \hline
    \end{tabular}
  \end{minipage}
  \newline
\end{center}

Tabel \ref{tab:tabeldua}, adalah contoh dari tabel yang memiliki banyak kolom.  Kita dapat melihat bagaimana LaTex dapat menangani tabel dengan sangat baik.

\end{document}
