\documentclass[12pt, times new roman]{article}
\linespread{1.5}
\usepackage{geometry}
\geometry{a4paper, total={210mm, 297mm}, left=40mm, right=30mm, top=30mm, bottom=30mm}
\usepackage{amsmath}
\begin{document}
Halo teman-teman,

Pada kesempatan kali ini kita akan belajar mengenai menulis equation matematika dengan LaTeX.

\begin{equation*}
  Luas = \frac{1}{2}\>at
\end{equation*}

\begin{eqnarray}
  (x - 5) (x + 6) & = & x^2 + 6x - 5x - 30 \\
                  & = & x^2 + x - 30
\end{eqnarray}

\begin{eqnarray}
  \begin{bmatrix}
    3  & 4  & 25 \\
    45 & 7  & 5  \\
    4  & 61 & 4
  \end{bmatrix}
\end{eqnarray}
\\
Menurut equation maka,

\begin{eqnarray}
  x = \left \{ \begin{array}{lr}
      Jika \> Genap & 0 \\
      Jika \> Ganjil & 1
    \end{array}\right.
\end{eqnarray}
\end{document}
